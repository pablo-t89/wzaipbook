\chapter{Wstęp}

% ------------------------------------------------------------------------------
\section{Motywacja do powstania tego skryptu}

Inspiracją do powstania tego skryptu były doświadczenia związane ze zdolnymi
studentami, którzy potrzebowali w krótkim czasie uzyskać pewne podstawowe
umiejętności programowania w języku \texttt{C++}. Studenci ci najczęściej
chcieli jak najszybciej zacząć brać aktywny udział w spotkaniach Sekcji
Algorytmicznej Koła Informatycznego Niepokoju ,,KINo'' \cite{sekcja}, w zajęciach przedmiotów
z serii ,,Wybrane Zagadnienia Algorytmiki i Programowania'' (WZAiP) \cite{wzaip} lub w zawodach
w~programowaniu zespołowym (na przykład tych wspomnianych w sekcji \ref{subsec:contests}). Na szczególne wyróżnienie jako inspiracja do nauki
technik programowania o których traktuje ten skrypt zasługują coroczne
Mistrzostwa Wielkopolski w~Programowaniu Zespołowym -- które są stacjonarne,
otwarte i przeprowadzane są na tak dużą skalę, że udział w nich może wziąć
praktycznie każdy zainteresowany takim doświadczeniem (więcej w sekcji \ref{subsubsec:mwpz}).

Styl programowania, o jakim traktuje ten skrypt jest związany z tradycjami
zawodów w stylu ACM-ICPC (ang. \emph{Association for Computing Machinery --
International Collegiate Programming Contest}) \cite{acm-icpc}. Dotyczy to wspomnianych
Mistrzostw Wielkopolski w Programowaniu Zespołowym oraz Algorytmicznych
Mistrzostw Polski w Programowaniu Zespołowym -- które z kolei są najbardziej
prestiżowym wydarzeniem tego typu w Polsce i stanowią lokalne eliminacje
do mistrzostw świata ACM-ICPC. Zadania z takich mistrzostw stanowią pewne
abstrakcyjne problemy, które należy rozwiązać poprzez dostarczenie kodu
źródłowego programu, który dla dowolnych danych wyznacza prawidłową odpowiedź,
przy zachowaniu ograniczeń czasowych oraz pamięciowych. Z pewnych praktycznych
(ale także tradycyjnych) przyczyn preferowany w tym zastosowaniu jest język
\texttt{C++} (\ref{sec:whycpp}). W istocie wykorzystywany jest jednak tylko
pewien podzbiór języka -- kody źródłowe rozwiązań mają charakter jednorazowy,
unikalny dla danego rozwiązania i tradycyjnie zapisywane są w całości
w pojedynczym pliku (\ref{subsec:whatdoweneed}). Takie wymagania mogą się wydać
egzotyczne nawet studentom informatyki, którzy zaliczyli przedmioty przecinające
się zakresem materiału z niniejszym skryptem (,,Algorytmy i Struktury Danych'',
,,Podstawy Programowania'' i ,,Programowanie Obiektowe'') -- dlatego
zawarty tu materiał zapewne będzie zawierał wiele nowych informacji także dla
większości z~nich.

Działalność dydaktyczna Sekcji Algorytmicznej często zawierała w sobie
wyjaśnianie pewnych podstaw podczas samych spotkań lub w ramach merytorycznych
paneli dyskusyjnych. Proponowane dotychczas źródła (\ref{sec:othersources}) nie
były jednak dostosowane do potrzeb uczestników idealnie -- rozwijało to w nich
bardzo cenną zdolność do samodzielnego wyszukiwania źródeł i oceny ich
przydatności, ale niestety także spowalniało sam proces kształtowania
podstawowych umiejętności programistycznych. Studenci Sekcji Algorytmicznej
wymieniają się często zestawem napisanych bardzo kolokwialnym językiem
,,czytanek'' proponowanych nowym uczestnikom. Niniejszy skrypt ma docelowo mieć
jednak nad dawnymi materiałami pewne zalety -- celem jest uzupełnienie luk
i sporządzenie materiału bardziej reprezentatywnego (w przeciwieństwie do
czytanek, które można określić jako hermetyczne i~nie mające racji bytu poza
kontekstem relacji towarzyskich Sekcji Algorytmicznej). Niniejszy skrypt ma
zatem być przydatny do celów promocji WZAiPów i~Sekcji kierowanej do nowych,
nie wdrożonych jeszcze w naszą barwną społeczność studentów.

W przypadku Wybranych Zagadnień Algorytmiki i Programowania problem tradycyjnie
nie był aż tak istotny, gdyż do udziału w przedmiocie zapraszani byli tylko
studenci otrzymujący odpowiednio dobry wynik podczas obowiązkowego testu
kompetencyjnego (wyjątki od tej reguły wymagały znacznego zaangażowania
i~wkładu pracy związanego z programowaniem) -- zapewniało to, że studenci
biorący udział w przedmiocie znaczną część podstaw opanowali w ramach pracy
samodzielnej. Drobne braki w wiedzy bez problemu uzupełniane były podczas zajęć
lub indywidualnych konsultacji. Często pozwalało to studentom, którzy uzyskali
w~ramach testu kompetencyjnego wynik na granicy przyjęcia na przedmiot dorównać
reszcie grupy i korzystać z uczestnictwa w przedmiocie w pełni.

Dodatkowe materiały często były potrzebne studentom matematyki, którzy chcieli
brać udział w wymienionych aktywnościach o charakterze programistycznym. Znaczna
część materiału WZAiPów (algebra, kombinatoryka i probabilistyka, geometria
obliczeniowa, teoria grafów, optymalizacja) jest łatwiejsza do zrozumienia dla
studentów kierunku matematyka -- często sprawia to, że po pewnym czasie
osiągnięcia studentów matematyki przewyższają średnią grupy z danego roku.
Opanowanie kwestii technicznych często sprawia jednak studentom matematyki
dodatkową trudność i wymaga od nich dodatkowego nakładu pracy w początkowych
tygodniach lub miesiącach WZAiPów. 


% ------------------------------------------------------------------------------
\section{Inne materiały}\label{sec:othersources}

\subsection{Tutoriale}

Próby sporządzenia zwięzłego wprowadzenia do języka \texttt{C++} można spotkać
w~wielu miejscach, ale ich jakość pozostawia wiele do życzenia. Bardzo ambitną,
lecz wiecznie nieukończoną próbą spisania takiego wprowadzenia jest
\mycite{wikibooks-cppprogramming}. Wprowadzeniem jeszcze bardziej zwięzłym,
a jednocześnie kompletnym jest \mycite{cpluspluscom-tutorial}. Źródła te mają
charakter ogólny i~nieznacznie wykraczają poza styl ACM-ICPC, ale są godne
polecenia ze względu na czytelną strukturę oraz niewielką długość.

\subsection{Podręczniki języka programowania}

Wskazanie obszernych źródeł książkowych do nauki języka programowania jest
zadaniem jeszcze trudniejszym. Jedną z możliwości jest rozpoczęcie nauki
programowania nie od języka \texttt{C++}, lecz od klasycznego \texttt{C} --
jest to dobra droga do uzyskania szerszego spojrzenia na temat oraz pod wieloma
względami optymalna kolejność poznawania tych języków. W tym celu można
śmiało korzystać z kanonu literatury jakim jest \mycite{kernighan1988}.

W przypadku samego języka \texttt{C++} analogicznym ,,kanonem'' literatury
jest \mycite{stroustrup2000}. Pozycję tę w~istocie polecam raczej teoretykom
języka, niż osobom zainteresowanym zaledwie umiejętnością posługiwania się
pewnym jego podzbiorem. Z całą pewnością pozycja ta nie jest kierowana do
czytelników, którzy wcześniej nie potrafili programować w~żadnym języku
imperatywnym.

Pozycją dalece bardziej przystępną jest \mycite{prata2004}. Jest to bardzo dobry
kompromis między pedantyczną starannością a przystępnym sposobem, w jaki opisany
został cały język \texttt{C++}.

Przykładem książki szczególnie wartościowej dla studentów Politechniki ze
względu na charakter zawartych w niej przykładów, a jednocześnie -- starannie
wprowadzającej we wszystkie elementy języka \texttt{C++} jest
\mycite{nyhoff2012}.

\subsection{Podręczna dokumentacja języka}

Referencje języka to materiały mające na celu nie tyle uczenie się nieznanych
elementów języka, co szybki dostęp do formalnego opisu znanych elementów
(nie oczekujemy wszakże, aby programista zawsze wszystko umiał na pamięć
-- programista powinien być twórczy, a brak zdolności odtwórczych nie musi
być wielką przeszkodą).

Znaczną formalną elegancją cechuje się referencja \mycite{cppreference}. Wiąże
się to jednak z pewnymi praktycznymi wadami -- referencja ta jest napisana
bardzo technicznym, suchym językiem i o ile ułatwia przypominanie sobie zasady
działania dowolnego elementu języka, to praktycznie uniemożliwia zrozumienie
jej od podstaw. Można przypuszczać, że taka forma jest niewygodna dla
programistów nie mających jeszcze wielkiego doświadczenia.

Dokumentacją sugerowaną dla początkujących programistów jest zatem
\mycite{cpluspluscom-reference}. Zapisana tam referencja języka \texttt{C++}
opatrzona jest zrozumiałym komentarzem i czytelnymi przykładami, które
skutecznie pobudzają wyobraźnię na temat tego, jak użyć każdego z omawianych
elementów języka.

\subsection{Podstawy teoretyczne}

W istocie przydatne podstawy teoretyczne mogłyby objąć lwią część programów
studiów na kierunkach matematyka i informatyka. Ta część skryptu będzie zatem
traktowała tylko o pozycjach najważniejszych, bądź wyróżniających się z~innego
powodu.

Temat potrzebnych podstaw matematycznych można w znacznej części sprowadzić do
zagadnień opisanych w \mycite{graham1994}.

Rozmowy na temat teorii grafów w języku polskim warto prowadzić posługując się
terminologią ze skryptu \mycite{filipczak2012}. Tłumacze najbardziej znanych
pozycji z dziedziny teorii grafów nie wypracowali języka, który byłby do końca
konsekwentny -- zaproponowany skrypt, poza wartościowym meritum, zawiera bardzo
rozsądne propozycje ujednolicenia polskich tłumaczeń wielu pojęć.

Wreszcie, jeśli chodzi o samą algorytmikę, warto przynajmniej raz w życiu
przeczytać \mycite{cormen2001} i wracać do tej pozycji przy każdej potrzebie.
Obszerność tej pozycji nie umniejsza temu, że jest ona obowiązkowa.

Przykładem bardziej przystępnej, pomocniczej pozycji traktującej o podstawach
algorytmiki jest \mycite{sedgewick1998}.

\subsection{Techniki programowania}

Pozycją traktującą o tym, jak programować dobrze jest przede wszystkim
\mycite{knuth1998} wraz z wciąć powstającymi dalszymi tomami.

Książką dotyczącą programowania na zawodach typu ACM-ICPC jest
\mycite{stanczyk2009}. Pewne sugestie na temat intensywnego zastosowania
preprocesora zapisane w tej książce polecam jednak traktować jako ciekawostkę
-- przepisywanie znacznej części języka w celu zaoszczędzenia kilku wpisywanych
znaków, które jednocześnie sprawia, że kod staje się niezrozumiały dla każdego
poza znającym konkretne makra autorem zdaje się być nieoptymalne.

\subsection{Zbiory zadań}

Klasycznym źródłem zbiorów zadań z omówieniami są związane z Olimpiadami
Informatycznymi ,,niebieskie książeczki'', czyli \mycite{oibook}.

Pozycjami -- o ile to możliwe -- jeszcze bardziej interesującymi są polskie
zbiory zadań, które w znacznej części traktują o zadaniach z zawodów
akademickich, czyli \mycite{diks2012} i \mycite{diks2015}.

\subsection{Tło kulturowe}

Znajomość materiałów takich jak \mycite{userfriendly}, \mycite{smbc},
\mycite{abstrusegoose} i~\mycite{xkcd} to absolutna podstawa i~nie wymaga 
dalszego komentarza.

W jednym ze źródeł wspomnianych powyżej można znaleźć odpowiedź na pytanie,
jak po upływie krótkiego czasu od rozpoczęcia nauki poznać język \texttt{C++}
\cite{abstrusegoose249} -- zaproponowane rozwiązanie jest znacznie
skuteczniejsze niż jakakolwiek propozycja bibliograficzna, nie wyłączając
niniejszego skryptu.

Ciekawym połączeniem meritum i humoru jest \mycite{green2000} -- pozycja,
która poza zapewnieniem dożywotniej pensji na stanowisku programisty w istocie
porusza wiele merytorycznych aspektów pisania czytelnego kodu oraz składni
języków programowania obiektowego, w tym \texttt{C++}.

% ------------------------------------------------------------------------------
\section{Weryfikacja osiągnięć}

Poznawanie tajników algorytmiki i programowania można rozpatrywać jako sztukę
dla sztuki. Jednak efekty nauki są bardziej mierzalne, jeśli znajdują one
odzwierciedlenie w rozwiązywanych problemach. Rozwiązywanie zadań w ramach
systemów typu online-judge jest ponadto źródłem pewnej satysfakcji, gdyż
w większości z nich dostępne są publiczne rankingi, w ramach których można
obserwować swoje postępy na tle pozostałych uczestników serwisu. Źródłem
znacznie większej satysfakcji jest jednak rozwiązywanie zadań w ramach
zawodów w programowaniu. Ta część skryptu traktuje o tym, gdzie można takie
zadania napotkać.

%TODO Do każdego serwisu i każdych zawodów dodać notę bibliograficzną z linkiem
%TODO niech Wojtek lub Olek napiszą o TopCoderze
\subsection{Ogólnodostępne systemy typu online-judge}

\subsubsection{SPOJ.com}

\subsubsection{Polski SPOJ}

\subsubsection{MAIN}

\subsubsection{Adjule}

\subsubsection{Codeforces}

\subsubsection{CodeChef}

\subsubsection{UVa Online Judge}

\subsubsection{HackerRank}

\subsubsection{CodinGame}

\subsubsection{CodeFights}


\subsection{Zawody w programowaniu}\label{subsec:contests}

\subsubsection{AMPPZ}

\subsubsection{MWPZ}\label{subsubsec:mwpz}

\subsubsection{PTwPZ}

\subsubsection{Deadline24}

\subsubsection{Marathon24}

\subsubsection{Potyczki Algorytmiczne}

\subsubsection{Google Code Jam}

\subsubsection{Facebook Hacker Cup}

\subsubsection{Rundy na Codeforces}

\subsubsection{CodeChef: wyzwania i Snackdown}

\subsubsection{Konkursy na CodinGame}

\subsubsection{Maratony na CodeFights}

