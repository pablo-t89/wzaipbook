% ------------------------------------------------------------------------------
\chapter{Podstawowe elementy języka}

\section{Wejście/wyjście}
% coś o buforowaniu
% endl, sync_with_stdio, tie
% ref do przeciążania operatorów

\section{Instrukcje warunkowe}
% if/else
% wspomnieć switch/case, pokazać jego wadę
% ref do zastępowania ifów mapą

\section{Pętle}
% break/continue nie zawsze boli. pokazać kiedy jest dobre, a kiedy złe.

\section{Funkcje}
% przekazywanie argumentów (wskaźnik wspomnieć i odradzić, poza przypadkiem
% przesuwania go w rekurencji)
% przekazywanie tablic, także wielowymiarowych statycznych (boli, więc może
% typedef, albo przejść na wektory wektorów)
% protypy, wsaźniki na funkcje

\section{Rekurencja}
% "czy jest limit na stos" (przykład: odwracanie napisu bez innej pamięci)
% + przekazywanie sobie referencji do czegoś co zapewni spamiętywanie
% (dać ref do stla i pokazać spamiętywanie w mapie)
% derekursywacja z własnym stosem (pokazać na tablicy, dać link do stack) 
